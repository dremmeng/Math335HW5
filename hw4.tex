% LaTeX Article Template - customizing page format
%
% LaTeX document uses 10-point fonts by default.  To use
% 11-point or 12-point fonts, use \documentclass[11pt]{article}
% or \documentclass[12pt]{article}.
\documentclass{article}

% Set left margin - The default is 1 inch, so the following 
% command sets a 1.25-inch left margin.
\setlength{\oddsidemargin}{0.25in}

% Set width of the text - What is left will be the right margin.
% In this case, right margin is 8.5in - 1.25in - 6in = 1.25in.
\setlength{\textwidth}{6in}

% Set top margin - The default is 1 inch, so the following 
% command sets a 0.75-inch top margin.
\setlength{\topmargin}{-0.25in}

% Set height of the text - What is left will be the bottom margin.
% In this case, bottom margin is 11in - 0.75in - 9.5in = 0.75in
\setlength{\textheight}{8in}
\usepackage{fancyhdr}
\usepackage{float}
\usepackage{mathtools}
\usepackage{amsmath}
\usepackage{amssymb}
\usepackage{graphicx}
\graphicspath{ {./} }

\setlength{\parskip}{5pt} 
\pagestyle{fancyplain}
% Set the beginning of a LaTeX document
\begin{document}

\lhead{Drew Remmenga MATH 335}
\rhead{HW \#4}
%\lhead{Independent Study}
%\rhead{R Lab}

\begin{enumerate}

\item 
	\begin{equation*}
	\begin{split}
f(y) & = (\theta+1)y^{\theta} \\
\mu = \int_{\mathcal{D}} f(y) dy &= \int_{0}^{1} (\theta+1)y^{\theta} y dy \\
 &= \int_{0}^{1} (\theta+1)y^{\theta+1} dy \\
&= \frac{(\theta+1)}{(\theta+2} y^{\theta +2} ]_{0}^{1} \\
&= \frac{(\theta+1)}{(\theta+2} \\
\bar{Y} &= \frac{(\theta+1)}{(\theta+2}\\
\bar{Y}(\theta +2) &= \theta+1\\
\bar{Y}\theta +2\bar{Y} &= \theta+1\\
\bar{Y}\theta - \theta &= 1 - 2\bar{Y} \\
\theta(\bar{Y} -1) & = 1 - 2\bar(Y)\\
\hat{\theta} & = \frac{1-2\bar{Y}}{\bar{Y} -1}
	\end{split}
	\end{equation*}
\item
	\begin{enumerate}
	\item
		\begin{equation*}
		\begin{split}
		l(\theta;X) = {100 \choose 33} \theta^{33}(1- \theta)^{100 -33} \\
		\end{split}
		\end{equation*}
	\item
		\begin{equation*}
		\begin{split}
		\mu &= n \theta \\
		\frac{\bar{X}}{100} &= \hat{\theta}
		\end{split}
		\end{equation*}
	\end{enumerate}
\item
	\begin{enumerate}
	\item
		\begin{equation*}
		\begin{split}
		\mu &= \frac{1}{\lambda}\\
		\bar{X} & = \frac{1}{\hat{\lambda}}\\
		\frac{1}{\bar{X}} & = \hat{\lambda}\\
		\end{split}
		\end{equation*}
	\item
		\begin{equation*}
		\begin{split}
		l(\lambda; X) & = \prod_{n=1}^{N} \lambda e^{-\lambda X_{n}} \\
		ln(l(\lambda; X) & =  ln(\prod_{n=1}^{N} \lambda e^{-\lambda X_{n}}) \\
		ln(l(\lambda; X) & =  N ln(\lambda) - \lambda \sum_{n=1}^{N} X_{n} \\
		\frac{d}{d\lambda} ln(l(\lambda; X) & = \frac{d}{d\lambda}[ N ln(\lambda) - \lambda \sum_{n=1}^{N} X_{n}] \\
		\frac{d}{d\lambda} ln(l(\lambda; X) & = \frac{N}{\lambda} - \sum_{n=1}^{N} X_{n} \\
		\frac{d}{d\lambda} ln(l(\lambda; X) & = 0 \\
		0 & = \frac{N}{\lambda} - \sum_{n=1}^{N} X_{n} \\
		\sum_{n=1}^{N} X_{n} & = \frac{N}{\lambda} \\
		\frac{\sum_{n=1}^{N} X_{n}}{N} & = \frac{1}{\lambda} \\
		\frac{1}{\bar{X}}  &= \hat{\lambda}
		\end{split}
		\end{equation*}
	\item
		\begin{equation*}
		\begin{split}
		\sqrt{n}(g(\hat{\lambda})-g(\mu)) & \rightarrow N(0, g'(\mu)^{2}\sigma^{2}) \\
		g'(\mu)   = -\mu^{-2} \\
		g'(\frac{1}{\lambda})  = -\lambda^{2} \\
		\sqrt{n} (\frac{1}{\bar{X}} - \lambda) &  \rightarrow N(0, \lambda^{2})\\
		\end{split}
		\end{equation*}
	\end{enumerate}
\item
	\begin{enumerate}
	\item
		\begin{equation*}
		\begin{split}
		P(X=1) = \lambda e^{-\lambda}
		\end{split}
		\end{equation*}
	\item
		\begin{equation*}
		\begin{split}
		\hat{MLE} & = \bar{X}e^{-\bar{X}}
		\end{split}
		\end{equation*}
	\item
		\begin{equation*}
		\begin{split}
		\sqrt{n} (\bar{X}e^{-\bar{X}} - \lambda e^{-\lambda}) &  \rightarrow N(0,\lambda(-\lambda e^{-\lambda}+e^{-\lambda})^{2} )\\
		\end{split}
		\end{equation*}
	\item
	$\lambda \neq 1$
	\item
		\begin{equation*}
		\begin{split}
		\hat{MLE} & = \bar{X}^{2} \\
		\end{split}
		\end{equation*}
	\item
		\begin{equation*}
		\begin{split}
		\sqrt{n} (\bar{X}^{2} - \lambda^{2}) &  \rightarrow N(0, 4 \lambda ^{3})\\
		\end{split}
		\end{equation*}
	\item
	$\lambda > 0 $
	\end{enumerate}
\end{enumerate}

\end{document}
